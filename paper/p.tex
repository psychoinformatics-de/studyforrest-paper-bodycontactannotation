\documentclass[10pt,a4paper,twocolumn]{article}
\usepackage{endfloat}
\usepackage{f1000_styles}
\usepackage{listings}
\usepackage{units}
\usepackage[colorlinks]{hyperref}
\usepackage{url}
\usepackage{appendix}
\usepackage{soul}

\definecolor{dkgreen}{rgb}{0,0.6,0}
\definecolor{gray}{rgb}{0.5,0.5,0.5}
\definecolor{mauve}{rgb}{0.58,0,0.82}

\begin{document}

%\input{results_def.tex}

\title{A comprehensive description of body contact depicted in the motion
picture ``Forrest Gump''}

\author[1]{Julia Coesfeld}
\author[1]{Helena Diedrich}
\author[1]{Juliane Hausmann}
\author[1]{Anna Kölpien}
\author[1]{Fiona Niebuhr}
\author[1]{Robert, Rak}
\author[1]{Kathleen Röder}
\author[1]{Pierre Ibe}
\author[1,2]{Michael Hanke}

\affil[1]{Psychoinformatics lab, Department of Psychology, University of Magdeburg, Universit\"{a}tsplatz 2, 39106 Magdeburg, Germany}
\affil[2]{Center for Behavioral Brain Sciences, Magdeburg, Germany}
\maketitle
\thispagestyle{fancy}

\begin{abstract}
% Abstracts should be up to 300 words and provide a succinct summary of the
% article. Although the abstract should explain why the article might be
% interesting, care should be taken not to inappropriately over-emphasise the
% importance of the work described in the article. Citations should not be used
% in the abstract, and the use of abbreviations should be minimized.




\end{abstract}
\clearpage

%The format of the main body of the article is flexible: it should be concise
%and in the format most appropriate to displaying the content of the article.

\section*{Introduction}
% Rationale for creating the dataset(s) and/or objectives for the experiment
% resulting in the dataset - why the data were gathered or produced.


\section*{Materials and methods}
% Detailed account of the protocol used to generate the dataset


\section*{Dataset validation}
% Information about any validation carried out and/or any limitations of the
% datasets, including any allowances made for controlling bias or unwanted
% sources of variability.


\section*{Data availability}

\texttt{This section will be auto-generated.}


\section*{Author contributions}
%In order to give appropriate credit to each author of an article, the
%individual contributions of each author to the manuscript should be detailed
%in this section. We recommend using author initials and then stating briefly
%how they contributed.

\section*{Competing Interests}
No competing interests were disclosed.

\section*{Grant Information}

Michael Hanke was supported by funds from the German federal state of
Saxony-Anhalt and the European Regional Development Fund (ERDF), Project: ,
Project: Center for Behavioral Brain Sciences.

\section*{Acknowledgements}
%This section should acknowledge anyone who contributed to the research or the
%article but who does not qualify as an author based on the criteria provided
%earlier (e.g. someone or an organisation that provided writing assistance).
%Please state how they contributed; authors should obtain permission to
%acknowledge from all those mentioned in the Acknowledgements section.  Please
%do not list grant funding in this section (this should be included in the
%Grant information section - See above).

%\nocite{*}
{\small\bibliographystyle{unsrt}
\bibliography{references}}

\end{document}

% vim: textwidth=80 colorcolumn=81
