\documentclass[10pt,a4paper,twocolumn]{article}
\usepackage{endfloat}
\usepackage{f1000_styles}
\usepackage{listings}
\usepackage{units}
\usepackage[colorlinks]{hyperref}
\usepackage{url}
\usepackage{appendix}
\usepackage{soul}

\definecolor{dkgreen}{rgb}{0,0.6,0}
\definecolor{gray}{rgb}{0.5,0.5,0.5}
\definecolor{mauve}{rgb}{0.58,0,0.82}

\begin{document}

\input{results_def.tex}

\title{A comprehensive description of body contact depicted in the motion
picture ``Forrest Gump''}

\author[1]{Julia Coesfeld}
\author[1]{Helena Diedrich}
\author[1]{Juliane Hausmann}
\author[1]{Anna Kölpien}
\author[1]{Fiona Niebuhr}
\author[1]{Robert Rak}
\author[1]{Kathleen Röder}
\author[1]{Pierre Ibe}
\author[1,2]{Michael Hanke}

\affil[1]{Psychoinformatics lab, Department of Psychology, University of Magdeburg, Universit\"{a}tsplatz 2, 39106 Magdeburg, Germany}
\affil[2]{Center for Behavioral Brain Sciences, Magdeburg, Germany}
\maketitle
\thispagestyle{fancy}

\begin{abstract}
% Abstracts should be up to 300 words and provide a succinct summary of the
% article. Although the abstract should explain why the article might be
% interesting, care should be taken not to inappropriately over-emphasise the
% importance of the work described in the article. Citations should not be used
% in the abstract, and the use of abbreviations should be minimized.

Body contact is an important aspect of everyday life, its observation
communicates versatile information about the haptic perception of its physical
aspects, but also about the intent and the emotional status of both the
initiator and the recipient. As such, the perception of body contact is highly
relevant for studying social interactions and the nature of its related
representations in the human brain. Recently, natural stimuli and, in
particular, motion pictures play an increasingly important role in studies
trying to explore the brain’s behavior under conditions of real-life
complexity. Several studies have investigated a publicly available brain
imaging dataset that used the two-hour motion picture “Forrest Gump” for
prolonged stimulation. A number of annotations for this stimulus have been made
available previously, including a description of portrayed emotions. Here we
extend the stimulus annotation with a detailed description of all body contact
events in the movie, in order to enable future studies on the representation
and interaction of physical and social aspects of body contact in natural
settings with near real-life complexity.


\end{abstract}
\clearpage

%The format of the main body of the article is flexible: it should be concise
%and in the format most appropriate to displaying the content of the article.

\section*{Introduction}
% Rationale for creating the dataset(s) and/or objectives for the experiment
% resulting in the dataset - why the data were gathered or produced.


Most experiments investigate the brain via simplified stimuli that cannot
reflect the complexity of a natural environment. With this project we want to
enable research to study the natural reaction of the human brain. In our
concrete case the body contact is in the spotlight. For observing the body
contact we used the movie „Forrest Gump“. The movie provides us complex
naturalistic stimuli representing real-life contexts better than laboratory
settings.

So far little scientific research regarding interpersonal touch/body contact
exists. Vastly more studies are concerned with the investigation of the
emotional aspects of vision and audition. (e.g. Ekman, 1993; Fecteau et al.,
2007).This is surprisingly because interpersonal touch is present in everyone's
daily life and plays an important role in interactions with other people. The
researcher Tiffany Field even asserts that touch affects us ten times stronger
than verbal or emotional context. (Field, 2001)  

Nevertheless are there some previous studies offering  us initial insights in
the subject.  Some of their  most important achievements concerning body
contact are for example the following ones.

Field (2001) and Spence (2002) detect that body contact plays an important role
in affect our emotional wellbeing. Interpersonal touch is capable of delivering
emotions and inspiring and motivating us sometimes more than words (Jones and
Yarbrough, 1985). An encouraging clap on the shoulder or a reconciling
handshake for example. 

In addition to this Hertenstein et al. (2006) investigated the way how we could
communicate with body contact. They detected that body contact could be used to
confide as many as six different types of emotion.

In other words body contact can be used to communicate emotion similar to that
in audition and vision.

Tiffany Field (2001) even assume that people in the modern society are
suffering from the absence or lack of interpersonal touch. This phenomenon
bears the significant name “touch hunger”.

Another interesting notice is that the eye contact whether or not we also touch
our counterpart at the same time can mean very different  things (Field, 2001).

Fischer et al. (1976) had a focus on an other aspect of the topic. The aim of
his studies was to consider the question whether interpersonal tactile
stimulation can influence others social behaviors regardless of the body
contact itself can be remembered consciously. According to his results it is
quite possible.

In summary it can be said that previous studies already provided us meaningful
cognitions but the research area for body contact is an ample field and because
of that still exist lots of open interesting questions and uninvestigated
sections.  One lack of research is for example the question how the tactile
aspects of tactile communication collude with visual and auditory aspects of
our surroundings. (Finnegan; 2005)

The aim of this study was to analyse the information contained in the movie
concerning the body contact for the purpose of facilitate further
investigations of the already published data. Furthermore we tried to get
insights in the topic of deriving and predicting the relationship of people by
means of their body contact.


\section*{Materials and methods}
% Detailed account of the protocol used to generate the dataset

\section*{Data legend}

For each annotated event, a total of 10 properties were recorded, each of which
are described in the following sections.

\paragraph{Start and end} The duration of each event is recorded in
\texttt{start} and \texttt{end} as the number of seconds from movie start (no
sub-second precision, due to limitations of the video player time display). The
time-points correspond to the onset and offset of the respective touch. Both
times can be identical in the case of events with less than one second duration.

\paragraph{Actor and recipient} The identity of a character initiating the body 
contact is encoded in \texttt{actor} using character labels listed in \citep{LRS+2015}.
The identify of the character being touched is encoded in the \texttt{sender} column.
In cases of interpersonal touch being initiated by both parties the event is listed twice with both characters being listed once as actor and recipient respectively.

\paragraph{Body part actor and body part recipient}

The \texttt{Bodypart_actor} and \texttt{Bodypart_recipient} variables contain space-seperated lists of body parts, involved in an event, for actor and recipient (see Table
\ref{tab:annotationcodes} for a list of possible body parts). 


\paragraph{label} The \texttt{label} flag assigns a label to the action that results in a body contact in a particular events. See Table
\ref{tab:annotationcodes} for a description of all possible labels.

\paragraph{Intensity of body contact} The \texttt{Intensity_of_body_contact} encodes the physical force with which actor and recipient touch eachother (1:~strong,
0:~weak).

\paragraph{Valence actor and valence recipient}
The \texttt{valence_actor} and \texttt{valence_recipient} variables encode how an inter-personal touch was rated emotionally be actor and recipient.
Ratings are measured on a four part Likert-scale ranging from "strong positive" to "strong negative" without a neutral category, thereby being forced-choice for the observer.

\paragraph{Intention} The \texttt{intention} variable encodes whether or not the body contact was initiated intentionally by the actor (1:~yes,
0:~no).

\paragraph{Audio information}
\texttt{audio_information} tracks whether there was additional information coming from the the movies audio track that points towards a body contact taking place. Audio cues such as noises were differenciated from parts of the narration directly speaking of an event. The variable is empty when no information is present. 

\section*{Dataset validation}
% Information about any validation carried out and/or any limitations of the
% datasets, including any allowances made for controlling bias or unwanted
% sources of variability. 
% This probably needs a "other" category for bodyparts and labels if we want to put those in aswell. 
% Valence could be done in a separate table with summs on the side, which would give additional information about valence and emotional arousal but I have to figure out how to do that in LaTex
% this entire table calc doesn't work yet because LaTex can't find the numbers for the variables. Maybe it has something to do with the underscore but I'm not sure.

\iffalse
\begin{table*}
  \centering
  \begin{tabular}{lrrr}
    \toprule\\
    & \multicolumn{3}{c}{\textbf{Event location min. agreement}} \\
    & \textbf{33\%} & \textbf{66\%} & \textbf{100\%} \\
    \\\midrule\\
    Number of events & \AggTwentyNEvents & \AggSixtyNEvents & \AggHundredNEvents \\
    Mean event duration (s) & \AggTwentyMeanEventDuration & \AggSixtyMeanEventDuration & \AggHundredMeanEventDuration \\
    Mean event distance (s) & \AggTwentyMeanEventDistance & \AggSixtyMeanEventDistance & \AggHundredMeanEventDistance \\
    \\\midrule\\
    \multicolumn{4}{l}{\textit{Body parts used by the actor}\\
    \multicolumn{4}{l}{\small[majority vote counts and inter-observer agreement (Fleiss' Kappa)]}\\\\
    Hand & \AggTwentyNbodypart_actorHand\ (\AggTwentyFKbodypart_actorHand) & \AggSixtyNbodypart_actorHand\ (\AggSixtyFKbodypart_actorHand) & \AggHundredNbodypart_actorHand\ (\AggHundredFKbodypart_actorHand)\\
    Arm & \AggTwentyNbodypart_actorArm\ (\AggTwentyFKbodypart_actorArm) & \AggSixtyNbodypart_actorArm\ (\AggSixtyFKbodypart_actorArm) & \AggHundredNbodypart_actorArm\ (\AggHundredFKbodypart_actorArm)\\
    Chest & \AggTwentyNbodypart_actorChest\ (\AggTwentyFKbodypart_actorChest) & \AggSixtyNbodypart_actorChest\ (\AggSixtyFKbodypart_actorChest) & \AggHundredNbodypart_actorChest\ (\AggHundredFKbodypart_actorChest)\\
    \\\midrule\\
    \multicolumn{4}{l}{\textit{Body parts of the recipients touched}}\\
    \multicolumn{4}{l}{\small[majority vote counts and inter-observer agreement (Fleiss' Kappa)]}\\\\
    Hand & \AggTwentyNbodypart_recipientHand\ (\AggTwentyFKbodypart_recipientHand) & \AggSixtyNbodypart_recipientHand\ (\AggSixtyFKbodypart_recipientHand) & \AggHundredNbodypart_recipientHand\ (\AggHundredFKbodypart_recipientHand)\\
    Arm & \AggTwentyNbodypart_recipientArm\ (\AggTwentyFKbodypart_recipientArm) & \AggSixtyNbodypart_recipientArm\ (\AggSixtyFKbodypart_recipientArm) & \AggHundredNbodypart_recipientArm\ (\AggHundredFKbodypart_recipientArm)\\
    Chest & \AggTwentyNbodypart_recipientChest\ (\AggTwentyFKbodypart_recipientChest) & \AggSixtyNbodypart_recipientChest\ (\AggSixtyFKbodypart_recipientChest) & \AggHundredNbodypart_recipientChest\ (\AggHundredFKbodypart_recipientChest)
    Face & \AggTwentyNbodypart_recipientFace\ (\AggTwentyFKbodypart_recipientFace) & \AggSixtyNbodypart_recipientFace\ (\AggSixtyFKbodypart_recipientFace) & \AggHundredNbodypart_recipientFace\ (\AggHundredFKbodypart_recipientFace)
    \\\\\midrule\\
    \multicolumn{4}{l}{\textit{Intensity of body contact}}\\
    \multicolumn{4}{l}{\small[majority vote counts and inter-observer agreement (Fleiss' Kappa)]}\\\\
    Strong & \AggTwentyNIntenseStrong\ (\AggTwentyFKIntenseStrong) & \AggSixtyNIntenseStrong\ (\AggSixtyFKIntenseStrong) & \AggHundredNIntenseStrong\ (\AggHundredFKIntenseStrong)\\
    Weak & \AggTwentyNIntenseWeak\ (\AggTwentyFKIntenseWeak) & \AggSixtyNIntenseWeak\ (\AggSixtyFKIntenseWeak) & \AggHundredNIntenseWeak\ (\AggHundredFKIntenseWeak)\\
    \\\\\midrule\\
    \multicolumn{4}{l}{\textit{Valence of actors}}\\
    \multicolumn{4}{l}{\small[majority vote counts and inter-observer agreement (Fleiss' Kappa)]}\\\\
    Strong Positive & \AggTwentyNValence_actorStrong_Positive\ (\AggTwentyFKValence_actorStrong_Positive) & \AggSixtyNValence_actorStrong_Positive\ (\AggSixtyFKValence_actorStrong_Positive) & \AggHundredNValence_actorStrong_Positive\ (\AggHundredFKValence_actorStrong_Positive)\\
    Positive & \AggTwentyNValence_actorPositive\ (\AggTwentyFKValence_actorPositive) & \AggSixtyNValence_actorPositive\ (\AggSixtyFKValence_actorPositive) & \AggHundredNValence_actorPositive\ (\AggHundredFKValence_actorPositive)\\
    Negative & \AggTwentyNValence_actorNegative\ (\AggTwentyFKValence_actorNegative) & \AggSixtyNValence_actorNegative\ (\AggSixtyFKValence_actorNegative) & \AggHundredNValence_actorNegative\ (\AggHundredFKValence_actorNegative)\\
    Strong Negative & \AggTwentyNValence_actorStrong_Negative\ (\AggTwentyFKValence_actorStrong_Negative) & \AggSixtyNValence_actorStrong_Negative\ (\AggSixtyFKValence_actorStrong_Negative) & \AggHundredNValence_actorStrong_Negative\ (\AggHundredFKValence_actorStrong_Negative)\\
    \\\\\midrule\\
    \multicolumn{4}{l}{\textit{Valence of recipients}}\\
    \multicolumn{4}{l}{\small[majority vote counts and inter-observer agreement (Fleiss' Kappa)]}\\\\
    Strong Positive & \AggTwentyNValence_recipientStrong_Positive\ (\AggTwentyFKValence_recipientStrong_Positive) & \AggSixtyNValence_recipientStrong_Positive\ (\AggSixtyFKValence_recipientStrong_Positive) & \AggHundredNValence_recipientStrong_Positive\ (\AggHundredFKValence_recipientStrong_Positive)\\
    Positive & \AggTwentyNValence_recipientPositive\ (\AggTwentyFKValence_recipientPositive) & \AggSixtyNValence_recipientPositive\ (\AggSixtyFKValence_recipientPositive) & \AggHundredNValence_recipientPositive\ (\AggHundredFKValence_recipientPositive)\\
    Negative & \AggTwentyNValence_recipientNegative\ (\AggTwentyFKValence_recipientNegative) & \AggSixtyNValence_recipientNegative\ (\AggSixtyFKValence_recipientNegative) & \AggHundredNValence_recipientNegative\ (\AggHundredFKValence_recipientNegative)\\
    Strong Negative & \AggTwentyNValence_recipientStrong_Negative\ (\AggTwentyFKValence_recipientStrong_Negative) & \AggSixtyNValence_recipientStrong_Negative\ (\AggSixtyFKValence_recipientStrong_Negative) & \AggHundredNValence_recipientStrong_Negative\ (\AggHundredFKValence_recipientStrong_Negative)\\
    \\\bottomrule
  \end{tabular}
  \caption{Annotation inter-observer agreement statistics.
  }
  \label{tab:validation}
\end{table*
\fi
\section*{Data availability}

\texttt{This section will be auto-generated.}


\section*{Author contributions}
%In order to give appropriate credit to each author of an article, the
%individual contributions of each author to the manuscript should be detailed
%in this section. We recommend using author initials and then stating briefly
%how they contributed.

\section*{Competing Interests}
No competing interests were disclosed.

\section*{Grant Information}

Michael Hanke was supported by funds from the German federal state of
Saxony-Anhalt and the European Regional Development Fund (ERDF), Project: ,
Project: Center for Behavioral Brain Sciences.

\section*{Acknowledgements}
%This section should acknowledge anyone who contributed to the research or the
%article but who does not qualify as an author based on the criteria provided
%earlier (e.g. someone or an organisation that provided writing assistance).
%Please state how they contributed; authors should obtain permission to
%acknowledge from all those mentioned in the Acknowledgements section.  Please
%do not list grant funding in this section (this should be included in the
%Grant information section - See above).

%\nocite{*}
{\small\bibliographystyle{unsrt}
\bibliography{references}}

\end{document}

% vim: textwidth=80 colorcolumn=81
